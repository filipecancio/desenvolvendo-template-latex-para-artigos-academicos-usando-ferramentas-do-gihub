\section{Conclusão e Trabalhos futuros}

O projeto realizado teve por motivação inicial a facilitação do desenvolvimento de artigos científicos no Instituto Federal de Educação Ciência e Tecnologia da Bahia, campus Vitória da Conquista. Atualmente o modelo SBC é utilizado em matérias como Trabalho de conclusão de curso, Metodologia de Pesquisa Científica e Inteligência de Negócio para a apresentação de atividades avaliativas. Com a utilização do projeto nessas atividades, alunos terão facilidade na produção de genêros acadêmicos e professores e orientadores poderão padronizar e ter mais controle sobre projetos de extensão e pesquisa no campus.

Para divulgar o projeto no campus, foi apresentado o minicurso ``Criando Artigos com LaTeX e GitHub'' durante a Semana de Tecnologia da Informação (Week-IT) edição de 2023. No minicurso, foram apresentados o template e os conceitos básicos para sua utilização. Com base na dinâmica e no retorno dos participantes, foi criada uma lista de reprodução no YouTube, que ensina passo a passo como utilizar o projeto na produção de gêneros acadêmicos.

Como sugestão para trabalhos futuros, o projeto pode receber melhorias como integração com a inteligência artificial da Microsoft, Copilot, para correções ortográficas em diferentes idiomas; também novas versões com templates específicos entre a faculdade,seja para os demais cursos de graduação, ou técnico. É possivel implementar o projeto para outras plataformas, como desenvolver o projeto usando o devcontainer nos produtos da Jetbrains (inteliJIDEA e Android Studio), isso permitiria que desenvolvedores Java e Android usassem o mesmo respositório de suas aplicações para desenvolver os seus artigos.

Além de explorar novas tecnologias, pode-se realizar futuramente um estudo para entender as dificuldades na elaboração de TCCs e outras monografias, bem como investigar de que maneira o projeto pode facilitar o aprendizado.

Espera-se que com esta ferramenta mais pessoas dentro do curso de Sistemas de Informação venham usar o LaTex para a produção de seus artigos, permitindo que autores e orientadores tenham mais controle sobre a produção de artigos científicos e que o Instituto possa ter um acervo virtual de artigos científicos produzidos por seus alunos e professores.