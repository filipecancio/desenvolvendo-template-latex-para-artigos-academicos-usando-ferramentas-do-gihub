\subsection{Metodologia}

Para realizar as atividades propostas foi necessário o planejamento de cada etapa. As ferramentas Project e Pull Requests do Github Auxiliaram na organização tanto durante a etapa de planejamento quanto durante as entregas de cada versão.

Todo o projeto foi planejado utilizando de método DevOps afim de realizar constantes entregas a cada nova mudança e acompanhar integração das novas funcionalidades. A utilização da ferramenta de Releases acompanhada da Actions foi crucial para cada integração e mudança realizada.


O desenvolvimento consistiu na criação de um ambiente de desenvolvimento utilizando Devcontainer, regras para gerar os arquivos ``.pdf'', testes, lint e uso do ambiente de forma online e offline e implementação do modelo LaTex de para formatação de textos.


A criações de regras para testes, lint para o Latex, e gerações de ``.pdf'' foram feitas utilizando a integração do projetos ``a-nau/latex-devcontainer'' e ``danteev/texlive'' que possui compilador e lint para o LaTex e ``xu-cheng/latex-action@v2'' para geração dos pdfs nas versões de release e desenvolvimento.

As versões de Devcontainer para o uso online foram configuradas com Codespaces e offline com Visual Studio Code e Docker.

Para a desenvolvimento da formatação do LaTex, foi utilizado o template oficial da SBC disponível em ``https://www.sbc.org.br/documentos-da-sbc/category/169-templates-para-artigos-e-capitulos-de-livros''. As informações em Latex usadas para a configuração de fonte, margens e demais formatações foram colocadas na pasta ``util'' com os arquivos $sbc.bst$ e $settings.sty$. Desta forma é possível usar o mesmo projeto sbs-template para a elaboração de outros tremplates em Latex apenas trocando as cofigurações destes arquivos.

Os testes de integração e lint foram executados durante o desenvolvimento porém a elaboração do artigo presente também fez parte da etapa de testes do projeto. Todo o artigo foi utilizado, para testar as funcionalidades, realizar melhorias e parte do conteúdo de apresentação e tutorial do projeto utilizou capturas de tela e gavações do artigo presente.