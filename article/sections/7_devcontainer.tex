

\subsection{Devcontainer}
Os compartimentos Docker são soluções de virtualização semelhantes às máquinas virtuais, porém mais enxutas e apenas com as informações necessárias para a sua execução~\cite{vitalino:01}. Para a criação de um container utilizamos uma imagem Docker, uma espécie de documento que declara as informações necessárias para determinado compartimento. Entre essas informações podemos citar por exemplo o sistema operacional, aplicativos que o compartimento deve conter, comandos iniciais ao criar o compartimento e conexões de rede local.
Os Devcontainers ou compartimento de desenvolvimento são compartimento Docker que possuem um ambiente completo com todas as configurações necessárias para o desenvolvimento de uma aplicação~\cite{github:01}. No repositório ``filipecancio/sbc-template'' usamos o Devcontainer para configurar todo o projeto latex, com o compilador para o ``.pdf'', ferramentas de edição e formatação do arquivo e uso de extensões do Visual Studio Code. Assim além de um projeto completo poder ser usado em uma maquina física com uma configuração automática, o mesmo pode ser facilmente removido sem muitas complicações em desinstalar o projeto e suas dependências.
Na figura~\ref{fig:image13}, podemos visualizar o repositório ``filipecancio/sbc-template'' em execução no Visual Studio Code e ao lado o compartimento em execução no docker. Podemos analisar o uso da CPU e memória do compartimento em tempo real.

\begin{figure}[ht]
	\centering
	\includegraphics[width=.5\textwidth]{./images/image13.png}
	\caption{Visualizando o devcontainer pelo Visual Studio Code e painel do Docker}
	\label{fig:image13}
\end{figure}

A implantação do Devcontainer consistiu nos diretórios ``.devcontainer'' e ``.vscode'', que respectivamente possui uma serie de arquivos de configuração para instalação de dependências e comandos para a execução dessas dependências no compartimento; e arquivos de configuração para o comportamento das dependências.
No diretório .devcontainer temos o arquivo devcontainer.json (figura~\ref{fig:image14}) que declara a imagem Docker pelo arquivo Dockerfile, a lista de extensões desejáveis e o script de inicialização do Devcontainer. O dockerfile possui uma imagem de referência ``danteev/texlive'' que cria um container Ubuntu com o compilador TexLive. O script ``initialize.sh'' inicia a configuração do Visual Studio Code e o CI (continuous integration) do container. As configurações de CI permitem que todas as vezes que o projeto é alterado, um novo arquivo PDF é gerado e sobrescreve o antigo na pasta article.

\begin{figure}[ht]
	\centering
	\includegraphics[width=.5\textwidth]{./images/image14.png}
	\caption{Arquivos dockerfile e devcontainer.json respectivamente}
	\label{fig:image14}
\end{figure}


