\section{desenvolvimento}

Durante o desenvolvimento foi elaborado o repositório chamado ``filipecancio/sbc-template'' utilizado como repositório template para o projeto. As soluções apresentadas para o projeto foram desenvolvidas utilizando de integração contínua, implementando primeiramente pipelines de verificação do Latex a cada salvamento de arquivo .Tex, a escolha de um arquivo .pdf alvo e posteriormente uma verificação final a cada commit gerado e submetido ao site do GitHub. Em paralelo foi utilizada da entrega contínua a geração automática de pdf e seus formatos de disponibilização. Ao total foram trës: Geração de pdf instantânea a cada arquivo salvo, Geração para versão de release, e geração de pdf a cada versão de pull request.
Para facilitação do uso do projeto para usuários comuns toda a parte escrita é restingida para a pasta article. Foi foi um manual de uso, presente no arquivo ``README.md'' do projeto, com um passo a passo completo.
Toda a implementação CI/CD (integração contínua, entrega contínua) foi implementada na elaboração do Devcontainer do projeto.